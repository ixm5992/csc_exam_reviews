%
% NOTE: This question is meant to take up one full page
%	and includes all necessary spacing.
%


\begin{enumerate}
\item What does the following function do?

\begin{lstlisting}
int foo(int n, int *arr, int **bestp)
{
	int *start;
	int *end;
	int best = 0;
	*bestp = arr;
	
	for (start = arr; start < arr + n; ++start)
	{
		for (end = start; end < arr + n && *end == *start; ++end);
		if ((end - start) > best)
		{
			best = (end - start);
			*bestp = start;
		}
		start = end - 1;
	}
	
	return best;
}
\end{lstlisting}

\begin{answer}
Finds the longest sequence of identical integers in the given array.
Returns the length of the sequence and stores the pointer to the start of the sequence in \texttt{bestp}.
\end{answer}

\item Make a memory map of \texttt{foo}. Use the first value set to each variable in the map.

\begin{answer}
Stack: n, arr, bestp, start, end, best

Heap: *arr (size n), *bestp (size int)

start \rightarrow arr

end \rightarrow start

bestp \rightarrow arr
\end{answer}

\item Given the following code in \texttt{main}, write the code to call our \texttt{foo} function above.
Also write the code to output the result.
\begin{lstlisting}[numbers=none]
int main(int argc, char **argv)
{
	int n, i;
	int arr[] = {1, 1, 1, 2, 2, 2, 5, 5, 5, 5};

	// Write your code here

	return 0;
}
\end{lstlisting}

\begin{answer}
\begin{lstlisting}
int main(int argc, char **argv)
{
	int n = 10;
	int arr[] = {1, 1, 1, 2, 2, 2, 5, 5, 5, 5};

	int *res = NULL;

	int i;
	n = foo(n, arr, &res);

	for (i=0;i<n;++i)
	{
		printf("%d, ", res[i]);
	}
	puts("");
	return 0;
}
\end{lstlisting}
\end{answer}
\end{enumerate}
