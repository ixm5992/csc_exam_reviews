Write a program that defines the structure of a queue:
\begin{itemize} {\small
	\item The queue ADT is described by its size (an \texttt{unsigned int}) and an array representing the queue's contents.
	\item The queue is built on an array of a specified size (think \texttt{\#define}), the elements of which are each described by their value (any data type).
	\item The queue's size describes the number of occupied indices in the array; in other words, the number of elements present within the queue.
	\item The queue will support the following operations:}
	\begin{itemize} {\scriptsize
		\item A function which returns an \texttt{int} representing whether or not the queue is empty, \texttt{int isEmpty(QueueADT q)}.
		\item A function which adds a new element (new data) onto the end of the queue, \texttt{void enqueue(QueueADT q, void* data)}.
		\item A function which removes the first element in the queue and returns it, \texttt{void* dequeue(QueueADT q)}.
		\item A function which returns an \texttt{unsigned int} representing the size of the queue, \texttt{unsigned int size(QueueADT q)}.}
	\end{itemize}
\end{itemize}

\begin{answer}
	\lstinputlisting[basicstyle=\scriptsize]{code/queue.c}
\end{answer}