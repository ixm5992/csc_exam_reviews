Explain the relationship between machine language, assembly language, and high level languages.\\
\begin{answer}
\small{\textbf{Machine language} is a set of instructions that is executed directly as-is by the computer's CPU itself. This is considered the lowest level representation of a computer program, and, while it is possible to program directly in machine code, it is highly tedious and error prone, making higher level languages favorable. Writing machine code is typically only done when troubleshooting a system or when implementing extreme optimization.\\
\textbf{Assembly language} is also considered a low-level programming language, and, in particular, assembly usually has a near 1:1 relationship between the assembly code and the architechure's machine code instructions. Assembly languages are specific to a computer's architechture - for example, assembly code you write for your phone (almost certainly) wouldn't work on your laptop. An example of an assembly language is MIPS. Programming in assembly language (and lower) is commonplace in embedded systems work.
\\
Finally, \textbf{high level languages} are, in general, designed to be portable across many different architechtures. What separates the high level languages from their lower level counterparts is that, due to this fact, they require compiling, which translates your high-level code into a form which the machine can understand (which varies by architechture.) When people talk about programming languages, they are most often referencing one of the high level languages, unless otherwise specified. High level languages include C and C++, Python, and Java.}
\end{answer}
