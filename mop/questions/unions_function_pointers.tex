\begin{enumerate}

\item Define a union, \texttt{typedef}ed to the name \texttt{printableData}, that can hold any of: an seven-character long string, a \texttt{double}, or an \texttt{int}.

\begin{answer}
\begin{verbatim}
typedef union {
    char stringData[8]; // 8 = 7 + NUL character
    double doubleData;
    int intData;
} printableData;
\end{verbatim}
\end{answer}

\item An idiom found in some C programs is the \textit{tagged union}.

\begin{verbatim}
typedef unsigned char tag_t;

// could also use an enum
const tag_t STRING_TAG = 0;
const tag_t DOUBLE_TAG = 1;
const tag_t INT_TAG = 2;

typedef struct {
    printableData data;
    tag_t tag;
} taggedData;
\end{verbatim}

In addition to the actual data, the tagged union stores a tag whose value corresponds to the type of data being stored.
Because of this, we can tell which of the three types of values are being stored in the \texttt{data} field.\\
Write a function that takes an array of \texttt{taggedData} and prints out the data each element holds, each on its own line.
Also take a \texttt{count} parameter.
Assume all necessary headers have been included.
(\textit{Hint:} Use a series of \texttt{if}/\texttt{else if} statements to properly print each type of data that can be stored.)

\begin{answer}
\begin{verbatim}
void printAll(taggedData* mixedArray, int count) {
    int i;
    for(i = 0; i < count; ++i) {
        if(mixedArray[i].tag == STRING_TAG) {
            printf("%s\n", mixedArray[i].data.stringData);
        } else if(mixedArray[i].tag == DOUBLE_TAG) {
            printf("%f\n", mixedArray[i].data.doubleData);
        } else if(mixedArray[i].tag == INT_TAG) {
            printf("%d\n", mixedArray[i].data.intData);
        }
    }
}
\end{verbatim}
\end{answer}

\item Suppose that we want to be able to process this data in other ways besides just printing to standard out. We can use function pointers to write a more generic \texttt{processAll} function. Suppose the following struct is defined:

\begin{verbatim}
typedef struct {
    void (*processString)(char*);
    void (*processDouble)(double);
    void (*processInt)(int);
} processingFunctions;
\end{verbatim}

Define a \texttt{processAll} function that takes an array of \texttt{taggedData},
a \texttt{count} parameter, and a \texttt{processingFunctions} value,
and uses the functions pointed to by the struct to process each element of the array,
rather than using \texttt{printf} as was used in the \texttt{printAll} function.

\begin{answer}
\begin{verbatim}
void processAll(taggedData* mixedArray, int count, processingFunctions procs) {
    int i;
    for(i = 0; i < count; ++i) {
        if(mixedArray[i].tag == STRING_TAG) {
            (*procs.processString)(mixedArray[i].data.stringData);
        } else if(mixedArray[i].tag == DOUBLE_TAG) {
            (*procs.processDouble)(mixedArray[i].data.doubleData);
        } else if(mixedArray[i].tag == INT_TAG) {
            (*procs.processInt)(mixedArray[i].data.intData);
        }
    }
}
\end{verbatim}
\end{answer}
\end{enumerate}

% http://ideone.com/l4z5kP
