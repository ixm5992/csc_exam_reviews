
Generally speaking, ADTs are easier to define and work with in procedural languages like C, as opposed to object-oriented languages like Java or C\#. (\textbf{True} or \textbf{False}). Explain your answer.

\begin{answer} 
\textbf{False}. C lacks object-oriented features that streamline the creation and use of ADTs in the language. C doesn't have access modifiers (e.g., \texttt{private} or \texttt{protected}), so hiding the underlying implementations is tougher, and will often involve the use of pointers.
\end{answer}

\item What is a \emph{primitive data type}? Provide 3 examples of primitive data types in C.

\begin{answer}
Primitive data types are supported by the language itself, as opposed to ones that are built from a combination of other objects (\emph{structured}), and ones that are extensions to the language (\emph{user-defined}). Examples include: \texttt{int}, \texttt{float}, \texttt{char}, \texttt{double}, \texttt{void}.
\end{answer}

\item Write a program that defines the structure of a queue:
\begin{itemize} {\small
	\item The queue ADT is described by its size (an unsigned int) and a pointer to the head/front of the queue.
	\item The queue is made up of nodes, which are each described by their value (any data type), and a pointer to the next node in the queue, or null if it is the last node in the queue.
	\item The queue's size describes the number of nodes present within the queue.
	\item The queue will support the following operations:}
	\begin{itemize} {\scriptsize
		\item A function which returns an int (0 or 1) representing whether or not the queue is empty, \texttt{int isEmpty(QueueADT q)}.
		\item A function which adds a new node onto the end of the queue, \texttt{void enqueue(QueueADT q, Node n)}.
		\item A function which removes the first node in the queue and returns it, \texttt{Node* dequeue(QueueADT q)}.
		\item A function which returns an unsigned int representing the size of the queue, \texttt{unsigned int size(QueueADT q)}.}
	\end{itemize}
\end{itemize}

\begin{answer}
	\lstinputlisting{code/queue.c}
\end{answer}


