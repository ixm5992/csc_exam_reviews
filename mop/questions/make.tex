Consider the following makefile:
\begin{lstlisting}
CFLAGS := -std=c99 -Wall -Wextra
me: me.o
    $(CC) $(LDFLAGS) -o $@ $^ $(LDLIBS)
calc: calc.o real.o
	$(CC) $(LDFLAGS) -o $@ $^ $(LDLIBS)
calc.o: calc.c
	$(CC) $(CFLAGS) -c $<
real.o: real.c
	$(CC) $(CFLAGS) -c $<
.PHONY: clean
clean:
	$(RM) me calc *.o
\end{lstlisting}

\begin{enumerate}
\item
Ralph's only friend, Jill, never defends him when the other children bully him.
Eventually, Jill becomes sick of his constant pleas for assistance and shouts, ``Why don't you make me?''
Always one to take things literally, Ralph pops open his favorite Bourne-compatible shell and types: \texttt{make me}.
However, he is confronted with the message: \texttt{make:\ 'me' is up to date}.
Explain the meaning of this message, being sure to state which (if any) of the relevant files are now located in Ralph's directory and to discuss what numeric values \texttt{Make} compared before outputting this message.

\begin{answer}
This output indicates that \texttt{Make} did not need to rebuild the \texttt{me} executable because:
\begin{itemize}
	\item the files \texttt{me.o} and \texttt{me} both existed already
	\item and the modification timestamp of \texttt{me.o} was earlier than that of \texttt{me}.
\end{itemize}
After the command completes, both files are still present in Ralph's directory.
The astute reader will note that, whatever her claim, Jill was already in existence throughout their discussion, and may be tempted to postulate that her inaction results not from her lack of being, but rather because she is not such a good friend after all.
\end{answer}

\item
Ralph is childishly proud of himself, but Jill just tells him to "get real".
Fortunately, Ralph has recently implemented a library providing functions for working with reals, as well as a calculator program to test the functionality.
Ready to win another trivial victory, Ralph types \texttt{make clean} and then \texttt{make calc}.
List the exact sequence of commands that are executed as a result of this new invocation.

\begin{answer}
Because the \texttt{clean} target eliminated all the object files and executables, Make will decide it needs to rebuild everything: \\
\texttt{\$ cc -std=c99 -Wall -Wextra -c calc.c} \\
\texttt{\$ cc -std=c99 -Wall -Wextra -c real.c} \\
\texttt{\$ cc -o calc calc.o real.o}
\end{answer}
\end{enumerate}
