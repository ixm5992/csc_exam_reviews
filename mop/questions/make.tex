Consider the following makefile:
\begin{lstlisting}
CFLAGS := -std=c99 -Wall -Wextra
me: me.o
    $(CC) $(LDFLAGS) -o $@ $^ $(LDLIBS)
calc: calc.o real.o
	$(CC) $(LDFLAGS) -o $@ $^ $(LDLIBS)
calc.o: calc.c
	$(CC) $(CFLAGS) -c $<
real.o: real.c
	$(CC) $(CFLAGS) -c $<
.PHONY: clean
clean:
	$(RM) me calc *.o
\end{lstlisting}

\begin{enumerate}
\item
Ralph's is annoying.  One day, when Ralph makes a particularly unreasonable demand, his friend loses it and shouts, ``Why don't you make me???''
Always one to take things literally, Ralph pops open his favorite Bourne-compatible shell and types: \texttt{make me}.
However, he is confronted with the message: \texttt{make:\ 'me' is up to date}.

Explain the meaning of this message.  Which (if any) of the relevant files are now located in Ralph's directory?  What numeric values did \texttt{Make} compare before outputting this message?

\begin{answer}
This output indicates that \texttt{Make} did not need to rebuild the \texttt{me} executable because:
\begin{itemize}
	\item the files \texttt{me.o} and \texttt{me} both existed already
	\item and the modification timestamp of \texttt{me.o} was earlier than that of \texttt{me}.
\end{itemize}
After the command completes, both files are still present in Ralph's directory.
\end{answer}

\item
Ralph is childishly proud of himself, but everyone just groans and tells him to "get real".
Coincidentally, Ralph has a library providing functions for working with reals, as well as a calculator program to test the functionality.
Ready to win another trivial victory, Ralph types \texttt{make clean} and then \texttt{make calc}.

List the exact sequence of commands that are executed as a result of this new invocation.

\begin{answer}
Because the \texttt{clean} target eliminated all the object files and executables, Make will decide it needs to rebuild everything: \\
\texttt{\$ cc -std=c99 -Wall -Wextra -c calc.c} \\
\texttt{\$ cc -std=c99 -Wall -Wextra -c real.c} \\
\texttt{\$ cc -o calc calc.o real.o}
\end{answer}
\end{enumerate}
