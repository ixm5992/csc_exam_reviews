Suppose we have the following bit field representing a player in a ripoff of the Halo video game series.

\begin{verbatim}
struct player {
    unsigned int is_alive:1;
    unsigned int team_color:1;
    unsigned int weapon_type:2;
    unsigned int ammo_remaining:4;
};
\end{verbatim}

However, you need to store these players as 8 bit \texttt{unsigned char} values.
You decide to store \texttt{ammo\_remaining} in the lowest 4 bits of the \texttt{unsigned char},
\texttt{weapon\_type} in the next 2 bits,
\texttt{team\_color} in the next bit,
and \texttt{is\_alive} in the highest bit.
Write a function that takes a \texttt{player} struct and
returns its \texttt{unsigned char} representation under the aforementioned rules.

\begin{verbatim}
unsigned char serializePlayer(struct player player1) {
\end{verbatim}
\begin{answer}
\begin{verbatim}
    // can also do a single return and `or` the values in one expression
    unsigned char data = 0;
    data |= player1.ammo_remaining;
    data |= player1.weapon_type << 4;
    data |= player1.team_color << 6; // 4 + 2
    data |= player1.is_alive << 7; // 4 + 2 + 1
    return data;
\end{verbatim}
\end{answer}
\begin{verbatim}
}
\end{verbatim}
