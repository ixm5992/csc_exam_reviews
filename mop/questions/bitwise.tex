Suppose we have the following bit field representing a player in an role-playing game.

\begin{verbatim}
struct player {
    unsigned int alive:1;
    unsigned int gender:1;
    unsigned int class:2;
    unsigned int level:4;
};
\end{verbatim}

However, you need to store these players as 8 bit \texttt{unsigned char} values. You decide to store \texttt{level} in the lowest 4 bits of the \texttt{unsigned char}, \texttt{class} in the next 2 bits, \texttt{gender} in the next bit, and \texttt{alive} in the highest bit. Write a function that takes a \texttt{player} struct and returns its \texttt{unsigned char} representation under the aformentioned rules.

\begin{verbatim}
unsigned char serializePlayer(struct player player1) {
\end{verbatim}
\begin{answer}
\begin{verbatim}
    // can also do a single return and `or` the values in one expression
    unsigned char data = 0;
    data |= player1.level;
    data |= player1.class << 4;
    data |= player1.gender << 6; // 4 + 2
    data |= player1.alive << 7; // 4 + 2 + 1
    return data;
\end{verbatim}
\end{answer}
\begin{verbatim}
}
\end{verbatim}
