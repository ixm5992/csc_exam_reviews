%
% NOTE: This question is meant to take up one full page
%	and includes all necessary spacing.
%


Below is a function that, given a lower bound \textit{i}, upper bound \textit{n}, and function from integers to integers \textit{f}, computes $\sum\limits_{k=i}^n f(k)$.
\begin{lstlisting}
	def series_sum(i, n, f):
		if(i == n):
			return f(n)
		return f(i) + series_sum(i + 1, n, f)
\end{lstlisting}
\begin{enumerate}
\item Invoke the function to compute $\sum\limits_{k=1}^5 k^2$.
\begin{answer}
\begin{multicols}{3}
\begin{verbatim}
	series_sum(1, 5, lambda x: x**2)
\end{verbatim}
\columnbreak
\hspace{80pt} OR
\columnbreak
\begin{verbatim}
	def square(x): return x**2
	series_sum(1, 5, square)
\end{verbatim}
\end{multicols}
\end{answer}

\item Rewrite the \texttt{series\_sum} function to be tail recursive.
\begin{answer}
\begin{lstlisting}
	def series_sum(i, n, f, s):
		if(i == n):
			return s + f(n)
		return series_sum(i + 1, n, f, s + f(i))
\end{lstlisting}
\end{answer}

\vspace{48pt}
\item What is the advantage of the new implementation? \\
\begin{answer}
Tail recursive functions make no additional calculations after the execution of the recursive call completes; thus, they do not necessitate the preservation of stack frames belonging to intermediate recursive invocations.
Provided the language implementation supports sufficient optimization, there is no more runtime memory overhead for calling a tail recursive function that makes 100 recursive calls as compared to one that makes none at all.
The true benefit becomes apparent when the recursion depth becomes much deeper: a tail recursive function can handle a recursion depth in the millions, which would cause a non--tail recursive one to overflow the stack.
\end{answer}
\end{enumerate}