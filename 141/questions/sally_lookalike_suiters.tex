%
% NOTE: This question is meant to take up one full page
%	and includes all necessary spacing.
%


Sally is being plagued by an army of lookalike suiters, each of which presents her with an enticing but unbearable meal upon arrival.
The dishes all smell amazing, however, so she can't help but try each one.
A dish will never fail to disappoint her, but fortunately, some of the suiters shared recipes and created identical concoctions.
Once Sally tastes a meal once, she can immediately smell out instances of the same dish and send their bearers away.
\begin{enumerate}
\item
\label{listdupes}
Write a Python function that, given a list of ``dishes,'' returns a list of the rejected dishes in the order they failed to fool her. (e.g. $[1, 2, 2, 3, 3, 3]$ would return [2,3,3] )
\begin{answer}
\begin{lstlisting}
def find_dupes(lst):
	result=[]
	seen=[]
	for member in lst:
		if member not in seen:
			seen.append(member)
		else:
			result.append(member)
	return result
\end{lstlisting}
\end{answer}

\item
If your solution to \ref{listdupes} was iterative, write it recursively; if it was recursive, write it iteratively.
\begin{answer}
\begin{lstlisting}
def find_recur(lst, seen=[]):
	if len(lst)==0:
		return []
	if lst[0] not in seen:
		seen.append(lst[0])
		return find_recur(lst[1:], seen)
	return [lst[0]]+find_recur(lst[1:], seen)
\end{lstlisting}
\textit{or}
\begin{lstlisting}
def find_tail(lst, result=[], seen=[]):
	if len(lst)==0:
		return result
	if lst[0] not in seen:
		seen.append(lst[0])
	else:
		result.append(lst[0])
	return find_tail(lst[1:], result, seen)
\end{lstlisting}
\end{answer}

\item
What is the time complexity of your approach?
Why? \\

\begin{answer}
$O(N^2)$, since the \texttt{seen} list must be searched linearly
\end{answer}
\end{enumerate}