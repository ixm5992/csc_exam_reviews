\begin{enumerate}

\item Write a binary tree program that defines the following:
\begin{list}{$\bullet$}{}
\item a \texttt{TreeNode} struct with the given properties:
	\begin{list}{-}{}
	\item \texttt{data} (an object) : the data in this node
	\item \texttt{left} : the \texttt{TreeNode} head of the left subtree (or \texttt{None})
	\item \texttt{right} : the \texttt{TreeNode} head of the right subtree (or \texttt{None})
	\end{list}
\item a \texttt{Tree} struct with the given properties:
	\begin{list}{-}{}
	\item \texttt{size} (an int) : the size of the binary tree
	\item \texttt{head} (a \texttt{TreeNode}) : the head node
	\end{list}
\item the appropriate maker functions for both structs
\item
	a recursive function \texttt{is\_in\_tree(head, element)} that determines
	whether the binary tree with the given head contains the specified element
\end{list}
\begin{answer}
	\lstinputlisting{code/binarytree.py}
\end{answer}

\item Give an example of how you could test your code.
\begin{answer}
\begin{lstlisting}[numbers=none]
newTree = mkTree(3, mkTreeNode(3, mkTreeNode(1, None, None),
		mkTreeNode(4, None, None)))
print(is_in_tree(newTree.head, 1)) # -> True
print(is_in_tree(newTree.head, 3)) # -> True
print(is_in_tree(newTree.head, 5)) # -> False
\end{lstlisting}
\end{answer}

\end{enumerate}
