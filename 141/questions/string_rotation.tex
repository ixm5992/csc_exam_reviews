Define a function that takes an input string and rotates the sequence
        of letters in each word by n. For example: shift\_left(``DEADBEEF'', 3)
        will produce the output string ``DBEEFDEA''. \\ shift\_left("Giant Robot", 4) will produce "t RobotGian"\\ 
        shift\_left("X", 5) will produce "X" \\You should be able to
        shift a string by a value greater than the length of the
        string\footnote{The modulus (\%) operator, which finds the remainder of a 
        division operation, will be useful here.}.\\ 
        \emph{Assume a function \texttt{len( str )} which returns the length of a string is provided.}

    \begin{enumerate}
        \item Design: Give brief description on how your function should
            accomplish this.

            \begin{answer}
            Our implementation will grab the first part of the new string by
            slicing all characters at an index after the offset. The second
            part of the string will be all of the characters up to the offset
            point. We will then concatenate the two strings.

            Making it possible for the offset to be greater than the length of
            the string can be accomplished by making {\tt offset = offset mod
            len(string)}
            \end{answer}

        \item Testing: Provide 3 test cases, using specific values for the input
            string and amount of shifting and what the expected output should be
            for each.

            \begin{answer}
                shift\_left(``'', 5) should return ""\\
                shift\_left(``A'',5) should return "A"\\
                shift\_left(``ABADCAFE'',300) == shift\_left("ABADCAFE",4)\\
                shift\_left(``FOO'', 1) should return "OOF"
            \end{answer}

        \item Implement the function in Python.

\begin{answer}
\begin{lstlisting}
def shift_left(string, offset):
    if len(string) == 0:
        return string
    else:
        offset = offset % len(string)
        first = string[offset:]
        last = string[:offset]
        return first + last
\end{lstlisting}
\end{answer}

        \item Implement the function {\tt shift\_right()}, which rotates letters
            in the opposite direction.

\begin{answer}
\begin{lstlisting}
def shift_right(string, offset):
    return shift_left(string, -offset)
\end{lstlisting}
\vspace{0.5in}
\end{answer}

    \end{enumerate}
