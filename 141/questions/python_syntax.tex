Python syntax\\
\emph{While most in the CS field will give a fair amount of leeway when it comes to coding by hand,
you nevertheless must have a reasonable level of familiarity with the language's syntax.}

\begin{enumerate}
\item Identify and fix the line(s) that have invalid syntax in the following code:\\
\emph{If the fix for a certain line isn't clear, an explanation of why the line is wrong is sufficient. For this question, assume that one line being incorrect doesn't affect the validity of other lines which may depend on it, where applicable.}
\begin{lstlisting}
include math

int a = 0
b = 0;
c = (int)(0)

for x in range(0, 100):
	print x

def function(int arg):
	return (arg + 2)

sum = b + math.sqrt(c)
total = a + (b ** 0.5)

myList = [1, 2, 3, 4, 5]
for element in myList:
	print(element)

for p in range(0, myList.len()-1):
	print(myList[p])
\end{lstlisting}
\begin{answer}
\begin{description}
	\item[1] The idea here is correct, but the line should read "import math", or, alternatively, "from math import *"
	\item[3] Python is a dynamically-typed language (as discussed in an earlier question.) You cannot specify a variable's type in this manner.
	\item[8] This is valid syntax in Python 2.x. In Python 3.x, 'print' is a function rather than a statement, thus you need parenthesis to specify arguments to the \texttt{print} function.
	\item[10] Like line 3, you cannot specify that a function argument is of a certain type, just that the function argument has a name associated with it.
	\item[20] You want to use the built-in Python statement \texttt{len()} like this:\\
	 "\texttt{for p in range(0, len(myList)-1):}".


\end{description}
\end{answer}

\item Which of the following are keywords or standard functions in Python?\\
\texttt{import\hspace{10mm}in\hspace{10mm}from\hspace{10mm}class\hspace{10mm}define\hspace{10mm}int\hspace{10mm}str\hspace{10mm}String}\\
\texttt{True\hspace{9mm}false\hspace{9mm}where\hspace{9mm}while\hspace{9mm}or\hspace{9mm}not\hspace{9mm}isinstance\hspace{9mm}new}\\
\texttt{print\hspace{13mm}range\hspace{13mm}float\hspace{13mm}char\hspace{13mm}bool\hspace{13mm}xor\hspace{13mm}sum}\\
\\
\begin{answer}
All of them except \texttt{define, String, false, where, new, xor, char}
\end{answer}
\end{enumerate}



