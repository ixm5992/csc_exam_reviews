In Quicksort, why should we select a random pivot value, rather than always
      pivoting on, for example, the first or last element?

      \begin{answer}
      With real-world data, we're more likely to encounter ordered or
      semi-ordered data than randomized data. This makes it more likely for us
      run into Quicksort's worst-case time complexity. We run into this bad
      time complexity if we select pivots which are near the lowest or highest
      values.

      Selecting a random value to pivot on helps us encounter the average case
      evens out the distribution of ordered and unordered data. Even if we're
      getting in sorted data, if we select pivots randomly, we should be able
      to end up with average time complexity.
      \end{answer}