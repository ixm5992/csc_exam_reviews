% Author: David Larsen <dcl9934@cs.rit.edu>
% Author: Doug Krofcheck <dpk3062@rit.edu>
\documentclass[11pt]{article}
\usepackage[margin=0.7in]{geometry}
\usepackage{listings}   %
\usepackage{needspace}  %
\usepackage{color}      %
\usepackage{ifthen}     % 
\usepackage{graphicx}   %
\usepackage{csc}        %
\usepackage{tikz}       %
\usetikzlibrary{shapes} %
\usepackage{tabularx}   % for helping matchtabular (matching questions)
\usepackage{textcomp}	% So our quotes in code don't look like shit
\usepackage{longtable}
\usepackage{multicol}
\setlength{\columnsep}{12em}

\lstset{ %
basicstyle=\footnotesize\ttfamily,       % the size of the fonts that are used for the code
numbers=left,                   % where to put the line-numbers
stepnumber=1,                   % the step between two line-numbers. If it's 1 each line will be numbered
numbersep=5pt,                  % how far the line-numbers are from the code
showspaces=false,               % show spaces adding particular underscores
showstringspaces=false,         % underline spaces within strings
tabsize=4,		                % sets default tabsize to 4 spaces
language=Java,
upquote=true,
columns=fixed
}

\ifthenelse{\isundefined{\isAnswerKey}}
{
    \newenvironment{answer}{\large\lstset{basicstyle=\large\ttfamily}\color{white} \small{Answer:}}{}
}
{
    \newenvironment{answer}{\large\lstset{basicstyle=\large\ttfamily}\color{red} \small{Answer:}}{}
}

% ----- Start matchtabular definition -----
\newcounter{matchleft}
\newcounter{matchright}
\newenvironment{matchtabular}{%
  \setcounter{matchleft}{0}%
  \setcounter{matchright}{0}%
  \tabularx{\textwidth}{%
    >{\leavevmode\hbox to 1.5em{\stepcounter{matchleft}\arabic{matchleft}.}}X%
    >{\leavevmode\hbox to 1.5em{\stepcounter{matchright}\alph{matchright})}}X% 
    }%
}{\endtabularx}
% ----- End matchtabular definition -----

\title{CSCI-142 Exam 2 Review}
\author{Computer Science Community}
\date{\today}

\makeatletter
\let\thetitle\@title
\let\theauthor\@author
\let\thedate\@date
\makeatother

\begin{document}
\header
\begin{enumerate}


\item Suppose we are talking about the depth-first search (DFS) algorithm.
\begin{enumerate}
\item What underlying data structure does this algorithm use?

\begin{answer}
A stack.
\end{answer}

\item %b
Given this graph, state the path that the DFS would generate and show the data structure at each step.
Assume nodes are \emph{added} to the data structure in alphabetical order.

\begin{answer}

\end{answer}
\end{enumerate}

\item Question \\
\begin{answer}
Answer
\end{answer}


\item Question \\
\begin{answer}
Answer
\end{answer}


\end{enumerate}
\end{document}

Topics for this exam:
	- Graphs: DFS
		#s
	- Graphs: BFS
		#s
	- Dijkstra's Shortest Path
		#s
	- Java Collection Framework, Iterators, Comparators
		#s
	- Intro. to GUIs (from OO perspective)
		#s
	- Swing
		#s
	- Exceptions
		#s

1) brief description of ?
2) 
3) 
4) 
6) 
7) 
8) 
9) 
