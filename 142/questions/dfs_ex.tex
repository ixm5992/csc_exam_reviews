Suppose we are talking about the depth-first search (DFS) algorithm.  Nodes are added to the data structures in alphabetical order.
\begin{enumerate}
\item What underlying data structure does this algorithm use?

\begin{answer}
A stack.
\end{answer}

\item
Given the following graph, state the DFS traversal order and show the data structure at each step.
Node \texttt{A} is the start node, and \texttt{F} is the destination node.

\begin{minipage}{0.35\textwidth}
\vspace{-136pt}
\begin{answer}
$\leftarrow$ top of stack \\
A \\
C B \\
E D B \\
D B \\
F B \\
B \\

The traversal order is ACEDF.
\end{answer}
\end{minipage}
\begin{tikzpicture}[shorten >=1pt,auto,node distance=2.8cm,semithick]
	\node[initial,state] (A) {A};
	\node[state]        (B) [above right of=A] {B};
	\node[state]        (C) [below right of=A] {C};
	\node[state]        (D) [right of=B] {D};
	\node[state]        (E) [right of=C] {E};
	\node[state]        (F) [below right of=D] {F};

	\path[->]   (A) edge node {} (C)
		(A) edge node {} (B)
		(B) edge node {} (C)
		(B) edge node {} (D)
		(C) edge node {} (D)
		(C) edge node {} (E)
		(D) edge node {} (F);
\end{tikzpicture}

\item
What path from \texttt{A} to \texttt{F} does the DFS algorithm return?

\begin{answer}
ACDF 
\end{answer}

\end{enumerate}