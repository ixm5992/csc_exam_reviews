Briefly explain the differences between the three kinds of exceptions: checked exceptions, runtime exceptions, and errors.
\begin{answer}

\textbf{checked exceptions} -� Exceptions that a method signature must specify it throws. If a method
may throw a checked exception, all calls to that method must be within a \texttt{try}-\texttt{catch} block. Checked exceptions should be used exclusively for foreseeable runtime mistakes, and any reasonably robust
system should be able to recover from one. Classic example is \texttt{IOException}.\\

\textbf{runtime exception} -� Not declared in a method signature and not anticipated to be thrown.
Usually arise due to software bugs and often cause the program to crash. Classic
examples are \texttt{NullPointerException} and \texttt{ArrayIndexOutOfBoundsException}.\\

\textbf{errors} -� Represent a serious issue outside of the control of the programmer (hard drive
failure, not enough memory, device issue). Examples are \texttt{IOError}, \texttt{VirtualMachineError} and
\texttt{ThreadDeath} (see Java's \texttt{Error} class).
\end{answer}


\vspace{48pt}