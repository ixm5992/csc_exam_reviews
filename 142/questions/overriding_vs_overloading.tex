What is the difference between overriding and overloading methods?  Give a example situation where each should be used. \\
\begin{answer}
\textbf{Overriding:} method with the exact same method declaration as a method in a superclass; overwrites and/or extends the functionality of the super() method.

\textbf{Overloading:} 2 or more methods with the same name and return type that either:
	\begin{enumerate}
	\item have a different number of parameters\\ \textbf{OR}
	\item have parameters of different types
	\end{enumerate}
Override a method when you want to \textit{replace} the implementation of a method in a superclass.  Overload a method (such as a constructor) when you want to have more than one implementation of a method for different circumstances.
\end{answer}