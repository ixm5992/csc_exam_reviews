Rick owns a positively popular pizza place conveniently located just off of campus.
	Originally, he made all the pizzas himself, but rising campus food prices are making demand skyrocket.
	Luckily, the college students are as desperate for money as they are for food, a situation from which Rick, being a pragmatic individual, finds he can benefit.
	Drawing from the exploitable labor pool, Rick turns his already-hot kitchen into a sweat shop, ordering his workers like so:
	\begin{lstlisting}[numbers=none]
	for ( PizzaSlave student : laborPool ) {
		new Thread(student) . run();
	}
	\end{lstlisting}
	Sensing an early retirement, Rick promotes his first hire from slave to manager, rewarding him with slightly higher---but still illegal---pay.
	(Despite these perks and the envy of his peers, the manager is just like everyone else.)
	Alas, when Rick returns a few days later, he is so displeased with what he sees that he fires the manager on the spot.
	What made Rick so angry, and what should he have done differently to prevent its happening?

	\begin{answer}
	In his haste to make a profit, Rick mistakenly called the \texttt{Thread} class's \texttt{run()} method, which caused his manager to run synchronously and make pizzas while everyone else stood around waiting.
	Rick should instead have called \texttt{start()}, which would have whipped all of his workers into shape at roughly the same time.
	\end{answer}