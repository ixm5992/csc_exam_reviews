\textbf{Searching a Graph}
	\begin{enumerate}
		\item
			Write a recursive algorithm that (given a graph, start vertex, and goal vertex),
			determines whether or not there is a path to the goal vertex.

            Assume you are provided with a \texttt{Graph} class with a \texttt{getNeighbors( int vertex )} method, which returns a \texttt{Set\textless Integer\textgreater} representing the numbers corresponding to neighboring vertices. Assume \texttt{visited} is a \texttt{Set} keeping track of all visited vertices. \\
			(Note: Your algorithm should return a Boolean value, not an actual path!)
\begin{verbatim}
boolean hasPathToRec(Graph g, int start, int goal, Set<Integer> visited) {
\end{verbatim}
			
\begin{answer}
\begin{lstlisting}
	if( start == goal ){
		return true;
	} else {
		for( int n : g.getNeighbors(start) ){
			if( ! visited.contains(n) ){
				visited.add(n);
				return hasPathToRec(g, n, goal, visited);
			}
		}
		return false;
	}
}
\end{lstlisting}
\end{answer}
		            
		\item
			Rewrite your algorithm to be iterative instead. \\
			(Hint: What data structure do you need to use if you no longer have recursion?)
\begin{verbatim}
boolean hasPathToIter(Graph g, int start, int goal, Set<Integer> visited) {
\end{verbatim}	
			
\begin{answer}
\begin{lstlisting}
	Stack<Integer> theStack = new Stack<Integer>();
	theStack.push(start);
	visited.add(start);
	while( ! theStack.empty() ){
		int curr = theStack.pop();
		if( curr == goal ){
			return true;
		}
		for( int n : g.getNeighbors(curr) ){
			if( ! visited.contains(n) ) {
				visited.add(n);
				theStack.push(n);
			}
		}
	}
	return false;
}
\end{lstlisting}
\end{answer}
	\end{enumerate}
