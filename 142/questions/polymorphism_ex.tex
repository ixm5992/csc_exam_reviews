Define polymorphism and explain a situation in which you would use it. \\
\begin{answer}
Polymorphism: when objects of various types define a common interface of operations (in other words, objects of different subclasses appear and act as a shared superclass). References to and collections of a super class may hold instances of subclasses. Methods invoked on these objects determine the correct (type-specific) behavior at runtime. That is, an object instantiated from a subclass will use the subclass's implementation of a method, even if it is stored in a reference to its superclass.

Example:

\begin{lstlisting}[basicstyle=\small]
ArrayList<Shape> shapes = new ArrayList<Shape>();

// Square with side length of 1
shapes.add(new Square(1.0));

// Circle with radius of 9
shapes.add(new Circle(9.0));

// Triangle with base of 4 and height of 5
shapes.add(new Triangle(4.0, 5.0));

// Outputs the correct area of each shape
for( int i = 0; i < shapes.size(); i++ ){ 
   	double area = shapes.get(i).getArea();
   	System.out.println(area);
}
\end{lstlisting}
\end{answer}
