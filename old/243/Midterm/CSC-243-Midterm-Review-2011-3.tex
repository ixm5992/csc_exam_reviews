\documentclass[11pt]{article}
\usepackage{fullpage}
\usepackage{listings}
\usepackage{needspace}
\usepackage{color}
\usepackage{ifthen}
\usepackage{pgf}
\usepackage{tikz}
\usetikzlibrary{arrows,automata}
\usepackage{amsmath}
\usepackage{csc}

\lstset{ %
basicstyle=\footnotesize,       % the size of the fonts that are used for the code
numbers=left,                   % where to put the line-numbers
stepnumber=1,                   % the step between two line-numbers. If it's 1 each line will be numbered
numbersep=5pt,                  % how far the line-numbers are from the code
showspaces=false,               % show spaces adding particular underscores
showstringspaces=false,         % underline spaces within strings
tabsize=4,		                % sets default tabsize to 4 spaces
language=Java
}

\ifthenelse{\isundefined{\isAnswerKey}}
{
    \newenvironment{answer}{\large\lstset{basicstyle=\large}\color{white}}{}
}
{
    \newenvironment{answer}{\large\lstset{basicstyle=\large}\color{red}}{}
}


\author{Computer Science Community}
\title{CS-243 Midterm Exam Review}
\date{\today}

\makeatletter
\let\thetitle\@title
\let\theauthor\@author
\let\thedate\@date
\makeatother

\begin{document}
\header

\begin{enumerate}

\item Why can we not use {\tt ==} to determine if two strings are equivalent?

    \begin{answer}
    {\tt ==} will check to see if two objects are the exact same object. If we
    want to compare objects by meaning, they'll have to implement the {\tt
    Comparable} interface, which declares the {\tt equals} method. For strings
    we could say {\tt ``yes''.equals(``yes'')}.
    \end{answer}

\item If an instance variable is declared with {\tt protected} access, who can
      access it?

    \begin{answer}
    Methods in that class and methods in all of its subclasses.
    \end{answer}

\item What is the difference between a class and an object?

    \begin{answer}
    A class is a construct that defines methods and properties - it can be viewed as a template. An objects are instances of that class. Think of classes as molds and objects as individual things created by those molds.
    \end{answer}

\item What is the difference between a constants and normal variables? Give an
      example of a constant declaration.

    \begin{answer}
    A constant may never change during run time (it may be set {\em once}).

    public static {\bf final} int numberOfPeopleWhoLikeJava = 0;
    \end{answer}

\item What is the difference between a constructor and a normal method?

    \begin{answer}
A Constructor is a special method which creates an instance of a class. It requires the use of the new keyword (or this() or super()) and can not be directly invoked by a method call. Constructors have no return type and have the same name as the class. 
    \end{answer}

\item What should the visibility (public, private, N/A) be on the following
      variables, constants and methods?\label{car-visibility}

\begin{lstlisting}
<visibility> class Car {
   <visibility> final String VIN;
   <visibility> String licensePlateNumber;  
   <visibility> String owner;

   <visibility> Car(String VIN, String licensePlateNumber, String owner) {
       this.VIN = VIN;
       this.licensePlateNumber = licensePlateNumber;
       this.owner = owner;
   }

   <visibility> void changeOwner( <visibility> String newOwner )
   {
       owner = newOwner;
   }
}
\end{lstlisting}

    \begin{answer}
\begin{lstlisting}
public class Car {
   public final String VIN;
   private String licensePlateNumber;  
   private String owner;

   public Car(String VIN, String licensePlateNumber, String owner) {
       this.VIN = VIN;
       this.licensePlateNumber = licensePlateNumber;
       this.owner = owner;
   }

   public void changeOwner( String newOwner )
   {
       owner = newOwner;
   }
\end{lstlisting}
    \end{answer}

\item Consider a class {\tt Car} that represents a car and stores information
      like the vehicle's license plate number, VIN, expiration date, owner's
      name etc.

    \begin{enumerate}
    \item Create a new object of class Car as defined in question
        \ref{car-visibility}.
        
        \begin{answer}
        Car myDepressingRide = new Car(``9270af8saf'', ``SAD MAN'', ``Elliot
        Smith'');
        \end{answer}

    \item Retreive the license plate number from the car object you just
    created by calling the getLicensePlate() method and store it in a variable.

        \begin{answer}
        String elliotsPlate = myDepressingRide.getLicensePlate();
        \end{answer}
    \end{enumerate}

\item Define the following terms:
    \begin{enumerate}
    \item Polymorphism

        \begin{answer}
        The ability to create an object of more than one type. References and collections of a super class may hold instances of subclasses. Methods invoked on these objects determine the correct (type specific) behavior at runtime. 
        \end{answer}

    \item Superclass

        \begin{answer}
A superclass is a class which is inherited from. A parent class.
        \end{answer}

    \item Subclass

        \begin{answer}
A subclass is a derived class which inherits one or more property from a parent or superclass. A child class.
        \end{answer}
    \end{enumerate}

\item What is the difference between an abstract class and an interface? Why
      might you want to use an interface over an abstract class?

      \begin{answer}
    Abstract classes allow for default method behavior to be defined, while interfaces allow only method declaration. Since java allows a class to extend only a single class but implement many interfaces, interfaces provide some extended capabilities. However, extending an abstract class allows usage of the defined default behaviors. 
     \end{answer}

\item When using overloaded methods, how does Java determine which method to
      call?

    \begin{answer}
The type and number of arguments determine which method to call.
    \end{answer}

\pagebreak
\item What gets printed by the following code?
\begin{lstlisting}
public class Class1 {
	public Class1() {
		System.out.println( "Class1()" );
	}
	public void print1() {
		System.out.println( "Class1.print1()" );
	}
	public void print2() {
		System.out.println("Class1.print2()" );
	}
}

public class Class2 extends Class1 {
	public Class2() {
		System.out.println( "Class2()" );
	}
	public void print1() {
		System.out.println("Class2.print1()");
	}
}

public class Class3 extends Class1{
	private Class2 class2;
	public Class3() {
		System.out.println( "Class3()" );
		class2 = new Class2();
	}
	public void print1() {
		class2.print1();
	}
	public void print2(){
		System.out.println("Class3.print2()");
		super.print2();
	}
}

public class TestClass {
	public static void main( String[] args ) {
		Class1 c1 = new Class2();
		c1.print1();
		c1.print2();

		System.out.println();
		Class1 c2 = new Class3();
		c2.print1();
		c2.print2();
	}
}
\end{lstlisting}
    
    \begin{answer}
Class1()\\
Class2()\\
Class2.print1()\\
Class1.print2()\\
\\
Class1()\\
Class3()\\
Class1()\\
Class2()\\
Class2.print1()\\
Class3.print2()\\
Class1.print2()\\

    \end{answer}

\pagebreak
\item Find at least 3 errors related to inheritance and interfaces in the
      following code:\label{demolition-derby}
\begin{lstlisting}
public interface Vehicle {
	public int getSpeed();
	public void accelerate(int speed_increase);
	public void brake(int speed_decrease);
}
public class Car implements Vehicle {
	private int speed;
	public Car(int initialSpeed){
		this.speed = initialSpeed;
	}
	public int getSpeed(){
		return speed;
	}
	public int accelerate(int speed_increase) {
		speed += speed_increase;
	}
	public int brake(int speed_decrease) {
		speed -= speed_decrease;
	}
}
public class Toyota extends Car {
	public long getSpeed(){
		return speed;
	}
	public void brake(Integer speed_decrease) {}
	public void factoryRecall(){
		System.out.println("Replace my floor mat!");
	}
}
public class Truck implements Vehicle {
	public void accelerate(int speed_increase) {
		super.accelerate(speed_increase/2);
	}
	public void brake(int speed_decrease) {
		super.brake(speed_decrease/2);
	}
}
public class demolitonDerby {
	public static void main(String[] args) {
		Vechicle	prius, mack, impreza;
		prius = new Toyota();
		mack = new Truck();
		impreza = new Car();
		
		impreza.accelerate(5);
		prius.brake(2);
		prius.factoryRecall();
		prius.accelerate(impreza.getSpeed());
		mack.accelerate(5.0);
	}
\end{lstlisting}

    \begin{answer}
    \begin{tabular}{r l} % TODO Fix some answers here
    Line \# & Error \\\hline
    14  & Returns int, should return void\\
    17  & Returns int, should return void\\
    22  & getSpeed()'s return type ({\tt long}) differs from Car or Vehicle.\\
    23  & speed is private.\\
    25,46& Calling brake() on Toyotas will call the superclass's brake()\\
    30  & Truck implements vehicle, but does not have getSpeed()\\
    32  & Can't call super(), this has no superclass.\\
    33  & Can't call super(), this has no superclass.\\
    47  & {\tt prius} was declared as a {\tt Vehicle}, so we can't call\\
    ~   & methods that aren't declared in {\tt Vehicle}.\\
    49  & FIXME I don't think it's legal to pass a double to an int.
    \end{tabular}
    \end{answer}

\end{enumerate}

\end{document}

Topics for this exam:
- Basic Java syntax
    - Method overloading
    - Visibility modifiers
    - The 'static' keyword
    - Generics
- Classes
- Interfaces
- Inheritance
